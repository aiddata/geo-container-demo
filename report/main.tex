\documentclass[a4paper]{article}

\usepackage[margin=1in]{geometry} 
\usepackage[utf8]{inputenc}
\usepackage{hyperref}
\usepackage{graphicx, color}
\usepackage{lipsum}
\usepackage{natbib}


% Research grant recipients must provide a summary report that is approximately two pages long, presenting the results of their funded research.  
% The report should include the status of the proposed work and note any significant changes from the initial proposal.
% The PI is expected to submit a summary report to the unit lead, Department Chair, Dean, Vice Provost for Research, and Provost.  

\title{Developing Data Pipelines for Research and Analysis}

\author{Seth Goodman$^1$\thanks{smgoodman@wm.edu} \and Jacob Hall$^1$ \and Cheyenne Hwang$^1$}

\date{
    $^1$AidData, Global Research Institute, William \& Mary \\ 
}

\begin{document}
\maketitle


\begin{abstract}

The growing demand for data and computational analysis in both research and educational initiatives stretches beyond disciplines such as computer or data science. As the efforts of faculty and students within research institutions become increasingly data-driven, traditional methods of utilizing critical resources such as high performance computing (HPC) clusters and disseminating the resulting work can present practical barriers to innovation. This work explores solutions to these issues through the development of a research use case which leverages open source software to build computational environments known as containers that can be deployed to run on computers ranging from individual laptops to HPC clusters or cloud-based clusters with minimal modification. We demonstrate the use of containers to deploy a replicable and scalable environment for running data pipelines to acquire, prepare, and analyze satellite based measures of vegetation around Chinese financed mining projects at seven sites around the world. The resulting prototype is now being expanded to support future research efforts and production data environments serving thousands of researchers around the world.

\noindent\textbf{Keywords:} containers, geospatial, data, replication, scalable
\end{abstract}


\section{Introduction}
	
The use case for the proposed work will focus on evaluating the impact of Chinese financed mining projects on vegetation levels in surrounding areas. The analysis will incorporate an automated data pipeline to acquire satellite imagery on vegetation levels, extract information around mining sites identified by AidData's Global Chinese Development Finance Dataset, and assess trends. The pipeline and analysis are intended to provide an illustrative and adaptable template for a wide range of potential computational tasks incorporating large scale data such as satellite imagery that could be built and deployed using containers. The ability to develop research concepts on one machine, scale up for analysis on a cluster, then distribute to others for replication and review - without having to modify, rebuild, or troubleshoot code - is essential to accelerating the integration of data and computational analysis across fields.


\section{Results}
\lipsum[1]

GeoQuery\citep{Goodman2019}
AidData's Geospatial Global Chinese Development Finance dataset\citep{Goodman2024}
NASA's LTDR NDVI\citep{NASA2023}
WorldPop\citep{WorldPop2018}


\section{Discussion}

Reference how this work served as the prototype for AidData's own scaled up data processing efforts using Kubernetes \footnote{See \url{https://github.com/aiddata/geo-datasets}}. Add more details on how it facilitated this, the amount of data we have processed, and how it has made it easier / supports GeoQuery (add in stats about GeoQuery usage, etc that this will help with).


\paragraph{Acknowledgements} 

We acknowledge the support of William \& Mary in funding this work through a W\&M Faculty Research Grant, and William \& Mary Research Computing for providing computational resources and/or technical support that have contributed to the results reported. 

\bibliographystyle{apalike}
\bibliography{refs.bib}	

\end{document}